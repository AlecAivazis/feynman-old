\documentclass{standalone}
\usepackage{feynman}

\begin{document}

\begin{feynman}
    \fermion[color=000000, endcaps=true, flip=false, label=l, labelDistance=-0.4, labelLocation=0.50, showArrow=true, lineWidth=1]{0, 4}{2, 2}
    \fermion[color=000000, endcaps=true, flip=false, label=\anti{l}, labelDistance=0.42, labelLocation=0.50, showArrow=true, lineWidth=1]{0, 0}{2, 2}
    \fermion[color=000000, endcaps=true, flip=false, label=l, labelDistance=-0.42, labelLocation=0.50, showArrow=true, lineWidth=1]{7, 4}{5, 2}
    \fermion[color=000000, endcaps=true, flip=false, label=\anti{l}, labelDistance=0.42, labelLocation=0.50, showArrow=true, lineWidth=1]{7, 0}{5, 2}
    \electroweak[color=55ABFF, endcaps=true, flip=false, label=\txt{Z}, labelDistance=0.42, labelLocation=0.50, showArrow=false, lineWidth=1]{2, 2}{5, 2}
\end{feynman}

\end{document}