\documentclass{standalone}
\usepackage{feynman}

\begin{document}

    \begin{feynman}[title=Example Feynman Diagram] 
        \gluon[color=000000, lineWidth=1, labelDistance=0.42, labelLocation=0.50, showArrow=true, flip=false, label=l, endcaps=false]{0, 4}{2, 2}
        \fermion[color=000000, lineWidth=1, labelDistance=0.42, labelLocation=0.50,color=FF0000, showArrow=true, flip=false, label=\anti{l}, endcaps=true]{0, 0}{2, 2}
        \fermion[color=000000, lineWidth=1, labelDistance=0.42, labelLocation=0.50, showArrow=true, flip=false, label=l, endcaps=true]{7, 4}{5, 2}
        \fermion[color=000000, lineWidth=1, labelDistance=0.42, labelLocation=0.50, showArrow=true, flip=false, label=\anti{l}, endcaps=true]{7, 0}{5, 2}
        \electroweak[lineWidth=1,label=Z,labelDistance=0.42,labelLocation=0.50,color=FF0000,showArrow=false,flip=false,endcaps=true]{2, 2}{5, 2}
   \end{feynman}

\end{document}