\documentclass{standalone}
\usepackage{feynman}

\begin{document}

    \begin{feynman}[title=Example Feynman Diagram] 
        \fermion[color=000000, endcaps=true, flip=false, label=l, labelDistance=0.42, labelLocation=0.50, showArrow=true, lineWidth=1]{2, 9}{0, 4} 
        \fermion[color=000000, endcaps=true, flip=false, label=\anti{l}, labelDistance=0.42, labelLocation=0.50, showArrow=true, lineWidth=1]{2, 0}{0, 4} 
        \fermion[color=000000, endcaps=true, flip=false, label=l, labelDistance=0.42, labelLocation=0.50, showArrow=true, lineWidth=1]{2, 9}{5, 7} 
        \electroweak[color=55ABFF, endcaps=true, flip=false, label=\txt{Z}, labelDistance=0.42, labelLocation=0.50, showArrow=false, lineWidth=1]{0, 4}{5, 7} 
        \fermion[color=000000, endcaps=true, flip=false, label=\anti{l}, labelDistance=0.42, labelLocation=0.50, showArrow=true, lineWidth=1]{2, 0}{5, 7} 
    \end{feynman}

\end{document}